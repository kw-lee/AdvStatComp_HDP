\section{Results}

\subsection{Data}
% 7. Simulation - generate문단은 송현, 

% 나머지는 결과 나오면 body맞춰서 알아서 채워넣기 MVN, CMVN, RCMVN, HCMVN, Block Reordering

\subsection{Multivariate Normal Probabilities} 

To implement \texttt{*MVN} functions, we need to calculate $n$-dimensional normal probability \eqref{eqn:normalprob},
$$
\Phi_n(a, b; 0, \Sigma) = \int_a^b \frac{1}{\sqrt{(2\pi)^n |\Sigma|}} \exp\left( -\frac{1}{2} \mathbf{x}^T \Sigma^{-1} \mathbf{x} \right) d\mathbf{x},
$$ 
numerically. We implement \texttt{mvn}, the function that calculate multivariate normal probabilities using Richtmyer Quasi-Monte Carlo(QMC) method proposed by \citet{genz2009computation}.

% table (table/qmc_vs_mc.ipynb)

It is well-known that QMC methods is more effective than classical Monte Carlo(MC) method. All the multivariate normal distribution probabilities required in the next algorithms are calculated using the \texttt{mvn} function.

\subsection{CMVN}
% CMVN/RCMVN - 송현 (table 2)

% \subsection{Cholesky Factorization}
% 하면야 좋겠지만..

\subsection{HMVN}
% HCMVN - 경원 (table 3)

In this section, we implement three methods in the section \ref{sec:hmvn} and compare theirs performance
\begin{itemize}
    \item \texttt{HMVN()}: Calculate multivariate normal probabilities using hierarchical-block approximation
    \item \texttt{HCMVN()}: Calculate multivariate normal probabilities using hierarchical-block conditioning approximation
    \item \texttt{HRCMVN()}: Calculate multivariate normal probabilities using hierarchical-block conditioning approximation with univarite reordering
\end{itemize}

\subsection{Block Reordering}
% Block Reordering - 현석 (table 5, 6)