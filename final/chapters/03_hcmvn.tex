
\section{Hierarchical-Block Approximations}\label{sec:hmvn}

In this section, we suggest methods to solve the $n$-dimensional MVN problem with the hierarchical covariance matrix using the $d$-dimensional conditioning method with that of the Monte Carlo-based method for solving the $m$-dimensional MVN problems presented by the diagonal blocks.
% 5. HCMVN - 경원 Hchol, CMVN, RCMVN, blocking

\subsection{Hierarchical Cholesky Decomposition}

\citet{hackbusch2015hierarchical} proposed adopting hierarchical matrix and its Cholesky decomposition. We have applied low rank approximation to each block of its decomposition to enhance computation efficiency and to save memory cost while accuracy is preserved.
$A=LU$have the structure
$$\begin{pmatrix}A_{11}&A_{12}\\A_{21}&A_{22}\end{pmatrix}=\begin{pmatrix}L_{11}&O\\L_{21}&L_{22}\end{pmatrix}\begin{pmatrix}L_{11}^T&L_{12}^T\\O&L_{22}^T\end{pmatrix}$$
with lower triangular matrix $L_{11},L_{22}$.
It comprises four tasks to complete the decomposition:
\begin{itemize}
	\item[(a)] compute $L_{11}$ via Cholesky decomposition of $A_{11}$
	\item[(b)] compute $L_{12}$ from $L_{21}L_{11}^T = A_{21}$
	\item[(c)] low rank approximation of $L_{12}=UV^T$
	\item[(d)] compute $L_{22}$ via Cholesky decomposition of $A_{22}-L_{21}L_{21}^T$
\end{itemize}
(a) and (d) are solved with hierachical Cholesky decomposition itself, and (b) is easy since it has triangular form. For (c), one needs to use low-rank approximation of SVD, i.e. $A=UDV^T=\sum_{i=1}^n d_i u_iv_i^T\approx\sum_{i=1}^k d_i u_iv_i^T$.
Algorithm for the hierachical Cholesky decomposition of $n\times n$ matrix into $m\times m$ blocks is stated below.
\begin{algorithm}[ht]
	\caption{Hierachical Cholesky decomposition}
	\begin{algorithmic}[1]
		\Procedure{\texttt{hchol}}{$A$, n,m,rank}
		\For{$i=1:log_2(\frac{n}{m})$}
		\State $nb = n/2^i$
		\State x = 0, y = nb
		\For{$j=1:2^{i-1}$}
		\State $\mathbf{U,D,V} = lowrankSVD(A[xbegin+1:xbegin+nb,ybegin+1:ybegin+nb], rank)$
		\State $\mathbf{A}[x + 1:x + nb, y + 1:y + rank] = \mathbf{UD}$
		\State $\mathbf{A}[x + 1:x + nb, y + rank+1:y + nb] = \mathbf{O}$
		\State $\mathbf{A}[y + 1:y + nb, x + 1:x + rank] = \mathbf{VD}$
		\State $\mathbf{A}[y + 1:y + nb, x + rank+1:x + nb] = \mathbf{O}$
		\State $x += 2nb, y += 2nb$
		\EndFor
		\EndFor
		\EndProcedure
		
	\end{algorithmic}\label{alg:hchol}
\end{algorithm}


\subsection{The Hierarchical-Block Conditioning Method}

Let $\phi_m(\mathbf{x}; \boldsymbol{\Sigma})$ be a pdf of $m$-dimensional normal distribution $N(\mathbf{0}, \boldsymbol{\Sigma})$ and $(\mathbf{B}, \mathbf{U}\mathbf{V}^T)$ be the hierarchical Cholesky decompostion of the covariance matrix $\boldsymbol{\Sigma}$. Then, we can express \eqref{eqn:normalprob} as 
\begin{equation}\label{eqn:hmvn}
    \Phi_n(\mathbf{a}, \mathbf{b}; \mathbf{0}, \boldsymbol{\Sigma}) 
    = \int_{\mathbf{a}_1'}^{\mathbf{b}_1'} \phi_m(\mathbf{x}_1; \mathbf{B}_1\mathbf{B}_1^T) 
    \cdots 
    \int_{\mathbf{a}_r'}^{\mathbf{b}_r'} \phi_r(\mathbf{x}_r; \mathbf{B}_r\mathbf{B}_r^T) d\mathbf{x}_r \cdots d\mathbf{x}_1.
\end{equation}
Where $\mathbf{a}',~\mathbf{b}'$, $i=1,\cdots,r$, are the corresponding segments of the updated $\mathbf{a}$ and $\mathbf{b}$. Specifficaly, we can compute $n$-dimensional MVN problem using hierarchical structure as algorithm \ref{alg:hmvn}.

\begin{algorithm}[ht]
    \caption{Hierarchical-block conditioning algorithm}
    \begin{algorithmic}[1]
    
    \Procedure{\texttt{HMVN}}{$a,~b,~\boldsymbol{\Sigma},~d$}
        \State $\mathbf{x} \leftarrow \mathbf{0}$ and $P \leftarrow 1$
        \State $[\mathbf{B}, \mathbf{UV}] \leftarrow$ \texttt{choldecomp\_hmatrix}$(\boldsymbol{\Sigma})$
        \For{$i = 1:r$}
            \State $j \leftarrow (i-1)m$
            \If{$i>1$}
                \State $o_r \leftarrow$ row offset of $\mathbf{U}_{i-1}\mathbf{V}_{i-1}^T$
                \State $o_c \leftarrow$ column offset of $\mathbf{U}_{i-1}\mathbf{V}_{i-1}^T$
                \State $l \leftarrow \dim(\mathbf{U}_{i-1}\mathbf{V}_{i-1}^T)$
                \State $\mathbf{g} \leftarrow \mathbf{U}_{i-1}\mathbf{V}_{i-1}^T\mathbf{x}[o_c+1:o_c+l]$
                \State $\mathbf{a}[o_r+1:o_r+l] = \mathbf{a}[o_r+1:o_r+l] - \mathbf{g}$
                \State $\mathbf{b}[o_r+1:o_r+l] = \mathbf{a}[o_r+1:o_r+l] - \mathbf{g}$
            \EndIf
            \State $\mathbf{a}_i \leftarrow \mathbf{a}[j+1:j+m]$
            \State $\mathbf{b}_i \leftarrow \mathbf{b}[j+1:j+m]$
            \State $P = P*\Phi_m(\mathbf{a}_i, \mathbf{b}_i; \mathbf{0}, \mathbf{B}_i\mathbf{B}_i^T)$
            \State $\mathbf{x}[j+1:j+m] \leftarrow \mathbf{B}_{i}^{-1} E(\mathbf{X}_i)$
        \EndFor
    \EndProcedure

    \end{algorithmic}\label{alg:hmvn}
\end{algorithm}

Note the probability value $\Phi_m(\mathbf{a}_i, \mathbf{b}_i; \mathbf{0}, \mathbf{B}_i\mathbf{B}_i^T)$ can be computed using QMC (\texttt{HMVN}, Method 1 in \citet{cao2019hierarchical}),  $d$-dimensional conditioning algorithm (\texttt{HCMVN}, Method 2 in \citet{cao2019hierarchical}) or with $d$-dimensional conditioning algorithm with univariate reordering (\texttt{HRCMVN}, Method 3 in \citet{cao2019hierarchical}). These methods are more effective and can easily be parallelized compared to the classical methods.

\subsection{Computational Complexity}

To compare computational complexity, we decomposed the complexity of Algorithm \ref{alg:hmvn} into three parts and listed the complexity for each part in Table \ref{tbl:cc_hmvn}, where $M(\cdot)$ denotes the complexity of the QMC simulation in the given dimension. 
%Table \ref{tbl:cc_hmvn} shows that the time efficiency of the $d$-dimensional conditioning algorithm mainly comes from lowering the dimension in which the QMC simulation is performed. 
% \renewcommand{\arraystretch}{1.5}
% \begin{table}[h]
%     \begin{center}
%         \begin{tabular}{l l l l}
%                             & MVN prob                     & Trunc exp          & Upd limits            \\
%             \hline
%             \texttt{HMVN}   & $\frac{n}{m} M(m)$           & $2nM(m) + O(nm^2)$ & $O(mn + kn log(n/m))$ \\
%             \texttt{HCMVN}  & $\frac{n}{d} M(d) + O(m^2n)$ & $2nM(d) + O(nd^2)$ & $O(mn + kn log(n/m))$ \\
%             \texttt{HRCMVN} & $\frac{n}{d} M(d) + O(m^2n)$ & $2nM(d) + O(nd^2)$ & $O(mn + kn log(n/m))$ \\
%             \hline
%         \end{tabular}
%         \caption{Complexity decomposition of the \texttt{HMVN}, \texttt{HCMVN}, and \texttt{HRCMVN}}\label{tbl:cc_hmvn}
%     \end{center}
% \end{table}
% \renewcommand{\arraystretch}{1}


\begin{table}[h]
	\centering
	% \resizebox{.6\textwidth}{!}{%
	{
		\begin{tabular}{@{}llll@{}}
            \toprule
                & MVN prob                     & Trunc exp          & Upd limits            \\
            \midrule
            \texttt{HMVN}   & $\frac{n}{m} M(m)$           & $2nM(m) + O(nm^2)$ & $O(mn + kn log(n/m))$ \\
            \texttt{HCMVN}  & $\frac{n}{d} M(d) + O(m^2n)$ & $2nM(d) + O(nd^2)$ & $O(mn + kn log(n/m))$ \\
            \texttt{HRCMVN} & $\frac{n}{d} M(d) + O(m^2n)$ & $2nM(d) + O(nd^2)$ & $O(mn + kn log(n/m))$ \\
            \bottomrule
		\end{tabular}%
	}
    \caption{Complexity decomposition of the \texttt{HMVN}, \texttt{HCMVN}, and \texttt{HRCMVN}}\label{tbl:cc_hmvn}
\end{table}

The three parts of the complexity are the calculation of the MVN probability (MVN prob), the calculation of the truncated expectations (Trunc exp), and the update of the integration limits with truncated expectations (Upd limits), respectively. The latter two share the same asymptotic order in all three complexity terms. The updating cost is independent to the method. The complexity of univariate reordering is $O(m^2 n)$, which is same as that of computing the MVN probabilities in \texttt{HCMVN}. Complexity from univariate reordering results in an identical major complexity component for \texttt{HCMVN} and \texttt{HRCMVN}. Since \texttt{HCMVN} and \texttt{HRCMVN} perform the QMC simulation in $d$-dimensions, the choice of $m$ does not affect the complexity of \texttt{HCMVN} and \texttt{HRCMVN}.
